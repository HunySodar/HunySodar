\documentclass{exam-zh}
\usepackage{siunitx}
%\usepackage{enumerate}
\usepackage{ulem}%字体下绘制各种线

\usepackage{xeCJKfntef}
\xeCJKsetup{underdot/symbol={\normalfont^^b7}}
\newcommand{\dotemph}[1]{\CJKunderdot{#1}}

\examsetup{
  page/size=a4paper,
  paren/show-paren=true,
  paren/show-answer=true,
  fillin/show-answer=false,
  solution/show-solution=false,
  question/show-points=true,% question 和 problem 环境中的分数全都显示
  choices/linesep=0em,
  material/format=\fzkai,
  poem/format=\fzkai
}
\setCJKfamilyfont {fzkaiti}{FZKTK.TTF}
\newcommand {\fzkai}{\CJKfamily{fzkaiti}}

\everymath{\displaystyle}

\title{2023年普通高等学校招生全国统一考试(新课标全国Ⅰ卷)}

\subject{语文}


\begin{document}

% \information{
%   姓名\underline{\hspace{6em}},
%   座位号\underline{\hspace{15em}}
% }
% \warning{(在此卷上答题无效)}

\secret

\maketitle

\begin{center}
    本试卷共 8 页,23 题。全卷满分 150 分。考试用时 150 分钟。
\end{center}



\begin{notice}
  \item 答题前,先将自己的姓名、准考证号、考场号、座位号填写在试卷和答题卡上,
    并将准考证号条形码粘贴在答题卡上的指定位置。
  \item 选择题的作答:每小题选出答案后,用 2B 铅笔把答题卡上对应题目的答案标号涂黑。
    写在试卷、草稿纸和答题卡上的非答题区域均无效。
  \item 填空题和解答题的作答:用黑色签字笔直接答在答题卡上对应的答题区域内。
    写在试卷、草稿纸和答题卡上的非答题区域均无效。
  \item 考试结束后,请将本试卷和答题卡一并上交。
\end{notice}



\section{%
现代文阅读(35分)
}

{\bfseries(一)现代文阅读I(本题共5小题,19分)}

阅读下面的文字,完成下面小题。

\begin{material}[source={(摘编自赫克托·麦克唐纳《后真相时代》,刘清山译)}]
对素食者和肠胃疾病患者来说,藜麦的发现是一个奇迹。藜麦不含麸质,富含镁和铁,比其他种子含有更多的蛋白质,包括人体无法独自生成的必需的氨基酸。美国宇航局宣布,藜麦是地球上营养最均衡的食物之一,是宇航员的理想之选。产于安第斯山的藜麦有一个令西方消费者神往的传说:印加人非常重视藜麦,认为它是神圣的,并且称之为“万谷之母”。不过,藜麦的爱好者却通过媒体发现了一个令人不安的事实。从2006年到2013年,玻利维亚和秘鲁的藜麦价格上涨了两倍。2011年,《独立报》称,玻利维亚的藜麦消费量“5年间下降了34%,当地家庭已经吃不起这种主食了,它已经变成了奢侈品”。《纽约时报》援引研究报告称,藜麦种植区的儿童营养不良率正在上升。2013年,《卫报》用煽动性标题提升了人们对这个问题的关注度:“素食者的肚子能装下藜麦令人反胃的事实吗?”该报称,贫穷的玻利维亚人和秘鲁人正在食用更加便宜的“进口垃圾食品”。《独立报》2013年一篇报道的标题是“藜麦:对你有利--对玻利维亚人有害”。这些消息传遍了全球,在健康饮食者之中引发了一场良心危机。在社交媒体、素食博客和健康饮食论坛上,人们开始询问食用藜麦是否合适。

这种说法看似可信,被许多人认可,但是经济学家马克·贝勒马尔等人对此则持保留意见。毕竟,藜麦贸易使大量外国资金涌入玻利维亚和秘鲁,其中许多资金进入了南美最贫穷的地区。几位经济学家跟踪了秘鲁家庭支出的调查数据,将种植且食用藜麦的家庭、食用但不种植藜麦的家庭和从不接触藜麦的家庭划分为三个小组。他们发现,从2004年到2013年,三个小组的生活水平都上升了,其中藜麦种植户家庭支出的增长速度是最快的。农民们正在变富,他们将这种新收入转化为支出又给周边民众带来了好处。那么藜麦消费量下降34%又是怎么回事呢?原来,在很长的时间内两个国家的藜麦消费量一直在缓慢而稳定地下降,这意味着消费量的下降和价格的激增不存在明显的联系。更加接近事实的解释是,秘鲁人和玻利维亚人只是想换换口味,吃点别的东西。

为了解藜麦的种植情况,我去了秘鲁科尔卡山谷,这里在印加时代以前就得到了开垦。藜麦是一种美丽的作物,拥有深红色或金黄色的巨大种球。在安第斯山的这片区域,人们在梯田上同时种植藜麦以及当地特有的玉米和马铃薯品种。“国外需求绝对是一件好事,”我的秘鲁向导杰西卡说道,“农民非常高兴,所有想吃藜麦的人仍然买得起这种食物。”她还解释了另一个好处。之前,秘鲁城里人往往认为他们这片区域吃藜麦的人“很土”。现在,由于美国人和欧洲人的重视,食用藜麦被视作一种时尚。“利马人终于开始尊重我们这些高原人和我们的传统了。”玻利维亚西南部有一片遥远而不适合居住的区域,那里到处都是盐湖和休眠火山。在那里,我看到了由藜麦资金支持的当地急需的开发和旅游项目。千百年来勉强能够养家糊口的自耕农开始为更加美好的未来而投资。我在2017年4月听到的玻利维亚人对于该作物的唯一抱怨是,日益增长的供给正在拉低价格。玻利维亚的藜麦种植面积增长了两倍多,从2007年的5万公顷增长到2016年的18万公项。马克·贝勒马尔后来对我说:“这是一个令人悲伤的结局,因为它的价格不太可能再度回升。”在风景如画的科尔卡山谷,当太阳落山时,我问杰西卡,欧洲和北美的消费者是否应该为吃掉秘鲁人和玻利维亚人的食物而感到内疚。我可以猜到答案,但我想听到当地人的亲口否认。“相信我,”杰西卡笑道,“我们有许多藜麦。”乍一看,这一关于食物热潮、全球贸易和消费者忧虑的事件讲述了谎言被揭穿的过程。不过,这些受到错误解读的真相可能会对当地的人们造成真正的伤害。各行各业有经验的沟通者会通过片面的事实、数字、背景呈现某种世界观,从而影响现实。在这个例子中,新闻工作者和博主出于高尚的理由引导消费者远离藜麦:他们由衷地为一个贫困群体感到担忧,害怕狂暴的全球贸易风潮会危及这一群体的利益。我们很早就知道这一点:每个新手辩论者和犯错误的小学生都知道如何挑选最有利于自己的真相。不过,我们可能不知道这些真相为沟通者提供了多大的灵活性。很多时候,你可以通过许多方式描述一个人、一件事物或者一起事件,这些描述可能具有同等的真实性。我将它们称为“竞争性真相”。

\end{material}

%1
\begin{question}[points = 3]
    下列对原文相关内容的理解和分析,不正确的一项是\paren
    
    \begin{choices}
        \item 藜麦适合素食者和肠胃疾病患者食用,并且由于其营养均衡,被美国宇航局认为是宇航员食物的理想之选。
        \item “马克·贝勒马尔等人对此则持保留意见”中的“此”,指的是被国外需求推高的藜麦价格给玻利维亚和秘鲁当地人造成伤害这一说法。
        \item 藜麦的大面积种植,不仅让玻利维亚和秘鲁等地农民的生活水平显著提高,而且改变了当地人对藜麦带有歧视的看法。
        \item 作者认为,尽管一些媒体引导消费者远离藜麦的做法值得商榷,但是其出发点却不是恶意的,他们为当地的贫困居民感到担忧。
    \end{choices}
\end{question}

%2
\begin{question}[points = 3]
    根据原文内容,下列说法不正确的一项是\paren

    \begin{choices}
        \item 从第一段的内容可以看出,当一起事件超出了我们直接观察的范围时,有些人会根据他人提供的信息,并结合自己的判断,生成对该事件的看法。
        \item “它的价格不太可能再度回升”,可能是因为市场对藜麦的需求量不再大幅增加,而藜麦的种植面积持续扩大,供给日益增长。
        \item “每个新手辩论者”“都知道如何挑选最有利于自己的真相”,可见有些“沟通者”会选择有助于推进个人意图的真相,而这种选择具有一定的灵活性。
        \item 从藜麦事件可以发现,一组片面的事实编织在一起引发了一场良心危机,而这场良心危机对玻利维亚和秘鲁当地的居民造成了真正的伤害。
    \end{choices}
\end{question}

%3
\begin{question}[points = 3]
    下列选项,最适合作为论据来支撑第二段观点的一项是\paren
    \begin{choices}
        \item “粮食优先”智库的工作人员塔尼娅·科森在谈到安第斯山藜麦种植者时表示:“坦率地说,他们厌倦了藜麦,因此开始购买其他食物。”
        \item 加拿大《环球邮报》一则新闻 标题为“你对藜麦的爱越深,你对玻利维亚人和秘鲁人的伤害就越深”。
        \item 制片人迈克尔·威尔科克斯专门为这个问题制作了一部纪录片,他说:“我见过一些反对食用藜麦的文字评论,实际上,停止消费才会真正伤害这些农民。”
        \item 英国广播公司播音员埃文·戴维斯指出:“事实上,说谎常常是没有必要的。你可以在不使用任何谎言的情况下完成许多有效的欺骗。”
    \end{choices}

\end{question}

%4
\begin{question}[points=5]
    请简要说明文本中的西方媒体在报道时使用了哪些“竞争性真相”。
\end{question}

%5
\begin{question}[points=5]
    作者采用哪些方法证明关于藜麦的新闻报道结论有误?请根据文本概括。
\end{question}

{\bfseries(二)现代文阅读II(本题共4小题,16分)}

阅读下面的文字,完成下面小题。

\begin{material}[title=给儿子,author=陈村,source=1984.8.5\\(有删改)
]
你总会长大的,儿子,你总会进入大学,把童年撇得远远的。你会和时髦青年一样,热衷于旅游。等到暑假,你的第一个暑假,儿子,你就去买票。

火车430公里,一直坐到芜湖。你背着包爬上江堤,看看长江。再没有比长江更亲切的河了。它宽,它长,它黄得恰如其分,不失尊严地走向东海。

你走下江堤,花一毛钱去打票,坐上渡船。船上无疑会有许多人。他们挑着担子,扛着被子,或许还有板车。他们说话的声音很高,看人从来都是正视。也许会有人和你搭话,你就老老实实说话。他们没有坏意。

你从跳板走上岸,顺着被鞋底和脚板踩硬踩白的大路,走半个小时。你能看到村子了。狗总是最先跳出来的。你可以在任何一家的门口坐下,要口水喝。主人总是热情的,而狗却时刻警惕着。也许会引来它的朋友们,纷纷表示出对你的兴趣。你要沉住气。

你谢过主人,再别理狗的讹诈,去河边寻找滩船。如果你运气好,船上只有一两个客,你就能躺在舱里,将头枕着船帮,河水拍击船底的声音顿时变得很重。船在桨声中不紧不慢地走。双桨“吱呀吱呀”的,古人说是“欸乃”,也对。怎么说怎么像。

板桥就在太阳落下去的地方。你沿着大埂走,右边是漕河,它连接着巢湖和长江。河滩如没被淹,一定有放牛的。你走过窑场就不远了。可以问问人,谁都愿意回答你,也许还会领你走一段,把咄咄逼人的狗子赶开。走到你的腿有点酸了,那就差不多到了。

走下大埂,沿着水渠边的路走。你走过一座小桥,只有一条石板的桥就是进村了。我曾写过它。这时,你抬起头,会发觉许多眼睛在看着你。

你对他们说,你叫杨子,你是我的儿子。

儿子,你得找和你父亲差不多年纪的人,他们才记得。

他们会记得那五个“上海佬”,记得那个戴近视眼镜的下放学生。他们会说他的好话和坏话。不管他们说什么,你都听着,不许还嘴。他们会告诉你一些细节,比如插不齐秧,比如一口气吃了个12斤的西瓜。你跟他们一起笑吧,确实值得笑上一场。

你们谈到黑了,会有人请你吃饭。不必客气,谁先请就跟谁去。能喝多少喝多少,能吃多少吃多少,这才像客人。天黑了,他们会留你住宿。他们非常好客。

儿子,你去找找那间草屋。它在村子的东头,通往晒场的路边,三面环水。你比着照片,看它还像不像当年。也许那草屋已经不在了,当年它就晃晃的,想必支撑不到你去。也许,那里又成了一片稻田。

晚上,你到田间小路上走走。你边走边读“稻花香里说丰年,听取蛙声一片”,感受会深深的。风吹来暖暖的热气,稻穗在风中作响。一路上,有萤火虫为你照着。

假如你有胆量,就到村东头的大坟茔去。多半会碰上“鬼火”,也就是磷火。你别跑,你坐在坟堆上,体会一下死的庄重和沉默。地下的那些人也曾生活在这块土地上,劳动,繁殖。他们也曾埋葬过他们的祖先。\underline{①你会捉摸到一点历史感的,这比任何教科书都有效。}

住上几天,你就熟悉村子了。男人爱理干干净净的发式,两边的头发一刀推净,这样头便显得长了。顶上则是长长的头毛,能披到眼睛,时而这么一甩,甩得很有点味道。

我喜欢见他们光着上身光着脚的样子。皮肤晒成了栗色,黑得发亮发光,连麦芒都刺不透它。他们不是生来这样的。和他们一起下河,你就知道,他们原先比你还白。现在,他们和你的祖先一样黑了。和你父亲当年一样黑。你要是下田,就和你一样黑。

下田去吧,儿子。让太阳也把你烤透。你弯下腰,从清晨弯到天黑,你恨不得把腰扔了。你的肩膀不是生来只能背背书包的。你挑起担子,肩上的肌肉会在扁担下鼓起。也许会掉层皮,那不算什么。你去拔秧,插秧,锄草,脱粒。你会知道自己并非什么都行。你去握一握大锹,它啥时候都不会被取代。工具越原始就越扔不了,像锤子,像刀,总要的。你得认识麦子,稻子,玉米,高粱,红薯。它们也是扔不了的。你干累了,坐在门边,看着猪在四处漫游,看着鸡上房,鸭下河,鹅窜进秧田美餐一顿。你听着杵声,感觉着太阳渐渐收起它的热力。你心平气和地想想,该说大地是仁慈的。它在无止无息地输出。我们因为这输出,才能存活,才得以延续。

那一层层茅草铺就的屋顶,那一条条小河分割的田野,那土黄色的土墙,那牛,那狗。那威力无比的太阳。

\underline{②你会爱的。}

你就这样住着,看着,干着。你去过了,你就会懂得父亲,懂得父亲笔下的漕河。当然,这实在不算什么,应当珍视的是你懂了自己。\underline{③你得不让自己飘了,你得有块东西镇住自己。}也许,借父亲的还不行,你得自己去找。

当你离开板桥的时候,人们会送你。你是不配的,儿子。你得在晚上告别,半夜就走。夜间的漕河微微发亮,你独自在河滩坐上一会,听听它的流动。

要是凑巧,你可以带条狗崽子回来。找条有主见的。开始,也许它有点想家。日子长了,你们能处好。你会发觉,为它吃点辛苦是值得的。

也就是这些话了,儿子。你得去,在大学的第一个暑假就去。\underline{④我不知道究竟会怎样。}要是你的船走进漕河,看见的只是一排烟囱,一排厂房,儿子,你该替我痛哭一场才是。虽然我为乡亲们高兴。

\end{material}

%6.
\begin{question}[points=3]
    下列对文本相关内容的理解,不正确的一项是\paren
    \begin{choices}
        \item 文章开头部分,父亲想象儿子上大学后会像时髦青年一样爱旅游,由此切入长大成人和出门旅行这两个关联话题。
        \item 儿子在渡船上会邂逅许多陌生人,父亲教给儿子,如何通过看他们的神情、听他们的言语来判断他们是否心存善意。
        \item 父亲设想儿子一路上常会遇到狗,并建议儿子离开时带走一条狗,可见狗应是父亲当年乡村生活中难忘的一部分。
        \item 儿子的板桥之旅除了坐车乘船,还需步行走过许多路,如江堤、大路、大埂、渠边小路、石桥等,带有较浓的寻访意味。
    \end{choices}
\end{question}

%7.
\begin{question}[points=3]
    对文中画线句子的分析与鉴赏,不正确的一项是\paren
    \begin{choices}
        \item 句子①中“你会捉摸到”的那种“历史感”,也正是“我”当年的经验和感悟。
        \item 句子②语义上与上段文字紧密相连,但单独成段,语气和表达的感情就更强烈。
        \item 句子③中的“飘”,是年轻人的一种心理状态,因脱离了父辈压制而感到飘然自在。
        \item 句子④表达出的不确定,与前文多处“你会”“你得”表现出的笃定形成了张力。
    \end{choices}
\end{question}

%8.
\begin{question}[points=5]
    “下田去吧,儿子”这个段落,写出了多重的身心感受。请加以梳理概括。
\end{question}

%9.
\begin{question}[points=5]
    读书小组要为此文写一则文学短评。经讨论,甲组提出一组关键词:未来·回忆·成长;乙组提出一个关键词:河流。请任选一个小组加入,围绕关键词写出你的短评思路。
\end{question}

\section{古代诗文阅读(35分)}

{\bfseries(一)文言文阅读(本题共5小题,20分)}

阅读下面的文言文,完成下面小题。

材料一:
\begin{material}[source=(节选自《韩非子·难一》)]
    襄子\textsuperscript{①}\dotemph{围}于晋阳中,出围,赏有功者五人,高赫为赏首。张孟谈曰:“晋阳之事,赫无大功,今为赏首,何也?”襄子曰:“晋阳之事,寡人国家危,社稷殆矣。\uline{吾群臣无有不骄侮之意者,唯赫子不失君臣之礼,是以先之。}”仲尼闻之,曰:“善赏哉,襄子!赏一人而天下为人臣者莫敢失礼矣。”或曰:仲尼不知善赏矣。夫善赏罚者,百官不敢侵职,群臣不敢失礼。上设其法,而下无奸诈之心。如此,则可谓善赏罚矣。襄子有君臣亲之泽,操令行禁止之法,而犹有骄侮之臣,是襄子失罚也。为人臣者,乘事而有功则赏。今赫仅不骄侮,而襄子赏之,是失赏也。故曰:仲尼不知善赏。
\end{material}

材料二:
\begin{material}[source=(节选自《孔丛子·答问》)]
    陈人有武臣,谓子鲋\textsuperscript{②}曰:“韩子立法,其所以异夫子之论者纷如也。予每探其意而校其事,持久历远,遏奸\dotemph{劝}善,韩氏未必非,孔氏未必得也。若韩非者,亦当世之圣人也。”子鲋曰:“今世人有言高者必以极天为称,言下者必以深渊为名。好事而穿凿者,必言经以自辅,援圣以自贤,欲以取信于群愚而度其说也。若诸子之书,其义皆然。\uline{请略说一隅,而君子审其信否焉。}”武臣曰:“诺。”子鲋曰:“乃者赵、韩共并知氏,赵襄子之行赏,先加\dotemph{具臣}而后有功。\uwave{韩非子云夫子善之引以张本然后难之岂有不似哉?}然实诈也。何以明其然?昔我先君以春秋哀公十六年四月己丑卒,至二十七年荀瑶与韩、赵、魏伐郑,遇陈恒而还,是时夫子卒已十一年矣,而晋四卿皆在也。后悼公十四年,知氏乃亡。此先后甚远,而韩非公称之,曾无怍意。是则世多好事之徒,皆非之罪也。故吾以是默口于小道,塞耳于诸子久矣。而子立尺表以度天,植寸指以测渊,矇大道而不悟,信\dotemph{诬说}以疑圣,殆非所望也。”
\end{material}

{\fangsong【注】①襄子:赵襄子。春秋末年,知、赵、韩、魏四家把持晋国国政,称“晋四卿”。晋阳之战,知氏(荀瑶)联合韩、魏攻赵,反被赵襄子联合韩、魏灭杀。②子鲋:即孔鲋,孔子八世孙。}

%10.
\begin{question}[points=3]
    材料二画波浪线的部分有三处需要断句,请用铅笔将答题卡上相应位置的答案标号涂黑,每涂对一处给1分,涂黑超过三处不给分。
\end{question}

韩非子\fbox{A}云夫子\fbox{B}善之\fbox{C}引\fbox{D}以张本\fbox{E}然\fbox{F}后难之\fbox{G}岂有\fbox{H}不似哉?

\begin{question}[points=3]
    下列对材料中加点的词语及相关内容的解说,不正确的一项是\paren
    \begin{choices}
        \item 围,指被围困,“傅说举于版筑之间”的“举”表示被选拔,两者用法相同。
        \item 劝,指鼓励、劝勉,与《兼爱》“不可以不劝爱人”中的“劝”词义不相同。
        \item 具臣,文中与“有功”相对,是指没有功劳的一般人臣,具体就是指高赫。
        \item 诬说,指没有事实依据 胡说妄言,与现在所说的“诬蔑之辞”并不一样。
    \end{choices}
\end{question}

\begin{question}[points=3]
    下列对材料有关内容的概述,不正确的一项是\paren

    \begin{choices}
        \item 主上设置有关法令,令行禁止,群臣不敢越职侵权,也没有了奸诈之心,他们履职行事,有了功劳就能得到赏赐,韩非认为这样才叫“善赏罚”。
        \item 在武臣看来,韩非与孔子观点不同的地方很多,在遏奸劝善等方面,韩非不一定就不对,孔子也不一定就合理,韩非也可以称得上是当世圣人。
        \item 世人说到高必定会以上天作比,说到低必定会以深渊作比,他们常通过引经据典、援用圣贤来成就自己,使自己更加贤能,以争取民众的信任。
        \item 子鲋对韩非之类的诸子学说闭口不言,充耳不闻,而武臣却深信不疑,进而怀疑圣人,子鲋对此深感失望,认为武臣是见识短浅,不明大道。
    \end{choices}
\end{question}

%13.
\begin{question}[points=8]
    把材料中画横线的句子翻译成现代汉语。
    
    \begin{enumerate}[label=(\arabic*)]
        \item 吾群臣无有不骄侮之意者,唯赫子不失君臣之礼,是以先之。
        \item 请略说一隅,而君子审其信否焉。
    \end{enumerate}
\end{question}

%14.
\begin{question}[points=3]
    子鲋用以批驳韩非的事实依据是什么?
\end{question}

{\bfseries(二)古代诗歌阅读(本题共2小题,9分)}

阅读下面这首宋诗,完成下面小题。
\begin{poem}[author=林希逸,title=答友人论学]
    逐字笺来学转难\zhu{\fangsong \zihao{5}笺:注释。这里指研读经典。},逢人个个说曾颜\zhu{曾颜:孔子的弟子曾参和颜回。}。\\
    那知剥落皮毛处,不在流传口耳间。\\
    禅要自参求印可,仙须亲炼待丹还。\\
    卖花担上看桃李,此语吾今忆鹤山\zhu{鹤山:南宋学者魏了翁,号鹤山。}。
\end{poem}

\begin{question}[points=3]
    下列对这首诗的理解和赏析,不正确的一项是\paren
    \begin{choices}
        \item 诗的首联描述了当时人们不畏艰难、努力学习圣人之道的学术风气。
        \item 诗人认为,“皮毛”之下精要思想的获得,不能简单依靠口耳相传。
        \item 颈联中使用“自”“亲”二字,以强调要获得真正学识必须亲自钻研。
        \item 诗人采用类比等方法阐明他的治学主张,使其浅近明白、通俗易懂。
    \end{choices}
\end{question}
%16. 
\begin{question}[points=6]
    诗的尾联提到魏了翁的名言:“不欲于卖花担上看桃李,须树头枝底方见活精神也。”结合本诗主题,谈谈你对这句话的理解。
\end{question}

{\bfseries(三)名篇名句默写(本题共1小题,6分)}

\begin{question}[points=6]
    补写出下列句子中的空缺部分。
    \begin{enumerate}[label=(\arabic*)]
        \item 司马迁在《报任安书》中说,自己编写《史记》“\fillin”,便遭遇了李陵之祸,因痛惜这部书不能完成,所以“\fillin”。
        \item 《旧唐书·音乐志》记载竖箜篌“体曲而长,二十有二弦”,而李贺《李凭箜篌引》中“\fillin,\fillin”两句,说明竖箜篌的弦数还有另一种可能。
        \item 小刚临摹了一幅诸葛亮的画像,想在上面题两句诗,却一直没想好。汪老师认为不妨直接用古人成句,比如“\fillin,\fillin”就很好。
    \end{enumerate}
\end{question}

\section{语言文字运用(20分)}
{\bfseries(一)语言文字运用I(本题共2小题,10分)}

阅读下面的文字,完成下面小题。

\begin{material}
    日常生活中,我们常常会因为忘记重要信息而懊恼,幻想着要是能过目不忘该多好啊!其实,我们更应该庆幸\uline{\hspace{2em}A\hspace{2em}},因为遗忘可以降低记忆带来的认知负荷,使认知系统能够更加高效地工作。而超强记忆力往往是以牺牲抽象、泛化能力为代价的。从下面例子中可以看出一些端倪。\par
    有一位记者,①拥有人们只能望其项背的超强记忆力。②他虽然能轻松地记住一长串数字,③却发现不了其中的规律;④他脑海里充满各种孤立的事实,⑤却不能归纳出一些模式将它们组织起来。⑥这促使他不能理解隐喻等修辞手法,⑦甚至复杂一点的句子。⑧记忆大师奥布莱恩曾多次获得世界记忆锦标赛冠军,⑨虽然他的阅读理解能力比常人低很多,⑩听课的时候也很难集中注意力。也许正是牺牲了一部分记忆,我们才有了独一无二的归纳和抽象思维能力。\par
    网络时代,我们没有办法也没有必要\uline{\hspace{2em}B\hspace{2em}},毕竟互联网随时可以帮我们查阅。\uwave{不过我们也不能过于依赖互联网,像互联网可以解决所有问题似的。}通过一些训练提升记忆力,也一直是我们孜孜以求的目标。

\end{material}

%18.
\begin{question}[points=5]
    请在文中画横线处补写恰当的语句,使整段文字语意完整连贯,内容贴切,逻辑严密,每处不超过10个字。
\end{question}

%19.
\begin{question}[points=5]
    文中第二段有三处表述不当,请指出其序号并做修改,使语言表达准确流畅,逻辑严密。不得改变原意。
\end{question}

{\bfseries(二)语言文字运用II(本题共3小题,10分)}

阅读下面的文字,完成下面小题。
\begin{material}
    天是越来越冷了,祥子似乎没觉到。心中有了一定的主意,眼前便增多了光明;在光明中不会觉得寒冷。地上初见冰凌,连便道上的土都凝固起来,\dotemph{处处}显出干燥,结实,黑土的颜色已\dotemph{微微}发些黄,像已把潮气散尽。特别是在一清早,被大车轧起的土棱上镶着几条霜边,小风尖溜溜的把早霞吹散,露出极高极蓝极爽快的天;祥子愿意\dotemph{早早}的拉车跑一趟,凉风飕进他的袖口,\uwave{使他全身像洗冷水澡似的一哆嗦,一痛快}。有时候起了狂风,把他打得出不来气,\uline{①可是他低着头,咬着牙,向前钻},像一条浮着逆水的大鱼;风越大,他的抵抗也越大,似乎是和狂风决一死战。猛的一股风顶得他透不出气,\uline{②闭住口,半天,打出一个嗝},仿佛是在水里扎了一个猛子。打出这个嗝,他继续往前奔走,往前冲进,没有任何东西能阻止住这个巨人;他全身的筋肉没有一处松懈,像被蚂蚁围攻的绿虫,全身摇动着抵御。这一身汗!等到放下车,直一直腰,吐出一口长气,抹去嘴角的黄沙,他觉得他是无敌的,他刚从风里出来,风并没能把他怎样了!
\end{material}

%20.
\begin{question}[points=3]
    文中有三个重叠形式“处处、微微、早早”,说说它们和“处、微、早”相比,语意上各自有什么不同。
\end{question}

%21.
\begin{question}[points=4]
    对文学作品来说,标点标示的停顿,有时很有表现力。文中有两处画横线部分,请任选一处,分析其中的逗号是怎样增强表现力的。
\end{question}

%22.
\begin{question}[points=4]
    语言文字运用I和II中画波浪线部分,都有“像……似的”,说说二者表意上的不同。

    \begin{enumerate}[label=(\arabic*)]
        \item 不过我们也不能过于依赖互联网,像互联网可以解决所有问题似的。
        \item 使他全身像洗冷水澡似的一哆嗦,一痛快。
    \end{enumerate}
\end{question}

\section{写作(60分)}

\begin{question}[points=60]
    阅读下面的材料,根据要求写作。
\end{question}

\begin{material}
好的故事,可以帮我们更好地表达和沟通,可以触动心灵、启迪智慧;好的故事,可以改变一个人的命运,可以展现一个民族的形象……故事是有力量的。
\end{material}

以上材料引发了你怎样的联想和思考?请写一篇文章。\par
要求:选准角度,确定立意,明确文体,自拟标题;不要套作,不得抄袭;不得泄露个人信息;不少于800字。


\end{document}
